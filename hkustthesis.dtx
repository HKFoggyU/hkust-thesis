% \iffalse meta-comment
% !TeX program  = XeLaTeX
% !TeX encoding = UTF-8
%
% Copyright (C) 2021 
% by <HKFoggyU> @ GitHub
% 
% It may be distributed and/or modified under the conditions of the
% LaTeX Project Public License (LPPL), either version 1.3c of this
% license or (at your option) any later version.  The latest version
% of this license is in the file
%
%    https://www.latex-project.org/lppl.txt
%
% -----------------------------------------------------------------------
%
% The development version of the template can be found at
%
%    https://github.com/HKFoggyU/hkust-thesis
%
% for those people who are interested.
%
%<*internal>
\iffalse
%</internal>
%
%<*internal>
\fi
\begingroup
  \def\NameOfLaTeXe{LaTeX2e}
\expandafter\endgroup\ifx\NameOfLaTeXe\fmtname\else
\csname fi\endcsname
%</internal>
%
%<*install>
\input l3docstrip.tex
\keepsilent
\askforoverwritefalse

\preamble

Copyright (C) 2021 
by <HKFoggyU> @ GitHub

This file may be distributed and/or modified under the conditions of
the LaTeX Project Public License, either version 1.3 of this license
or (at your option) any later version.  The latest version of this
license is in:

   http://www.latex-project.org/lppl.txt

and version 1.3 or later is part of all distributions of LaTeX version
2005/12/01 or later.

To produce the documentation run the original source files ending with `.dtx'
through XeTeX.
    
\endpreamble

\generate{
  \usedir{tex/latex/hkustthesis}
    \file{\jobname.cls}        {\from{\jobname.dtx}{class}}
%</install>
%<*internal>
  \usedir{source/latex/hkustthesis}
    \file{\jobname.ins}        {\from{\jobname.dtx}{install}}
%</internal>
%<*install>
}

\obeyspaces
\Msg{**************************************************************}
\Msg{*                                                            *}
\Msg{* To finish the installation you have to move the following  *}
\Msg{* files into a directory searched by TeX:                    *}
\Msg{*                                                            *}
\Msg{* The recommended directory is TDS:tex/latex/hkustthesis     *}
\Msg{*                                                            *}
\Msg{*     hkustthesis.cls                                        *}
\Msg{*     hkustthesis.ins                                        *}
\Msg{*                                                            *}
\Msg{* To produce the documentation, run the file hkustthesis.dtx *}
\Msg{* through XeLaTeX.                                           *}
\Msg{*                                                            *}
\Msg{* Happy TeXing!                                              *}
\Msg{*                                                            *}
\Msg{**************************************************************}

\endbatchfile
%</install>
%
%<*internal>
\fi
%</internal>
%
%<class>\NeedsTeXFormat{LaTeX2e}
%<class>\RequirePackage{expl3}
%<class>\GetIdInfo  $Id: hkustthesis.dtx 0.1 2021-10-21 00:00:00 +0800  HKFoggyU$
%<class>  { Thesis template for HKUST }
%<class>\ProvidesExplClass{hkustthesis}
%<class>{\ExplFileDate}{\ExplFileVersion}{\ExplFileDescription}
%
%<*driver>
\ProvidesFile{hkustthesis.dtx}
\documentclass[12pt]{ctxdoc}
\usepackage{listings,xcolor,tabularray}
\setlist[1]{labelindent=0.5em}
\UseTblrLibrary{booktabs,siunitx,diagbox}
\DefTblrTemplate{caption-tag}{default}{Table~\hspace{0.25em}\thetable}
\SetTblrStyle{caption-tag}{font=\bfseries}
\DefTblrTemplate{caption-sep}{default}{\quad}
\definecolor{hkustblue}{RGB}{0, 51, 102}
\definecolor{hkustgold}{RGB}{153, 102, 0}
\definecolor{hkustred}{RGB}{237, 27, 47}
\definecolor{hkustgray}{RGB}{204, 204, 204}
\definecolor{hkustviolet}{RGB}{124, 35, 72}
\definecolor{hkustyellow}{RGB}{255, 212, 0}
\begin{document}
  \DocInput{hkustthesis.dtx}  
  %\PrintChanges
  %\PrintIndex
\end{document}
%</driver>
% \fi
%
% \title{\color{hkustblue}{The \textsc{HkustThesis} class\\ A \hologo{LaTeX}\textcolor{hkustred}{3} template}}
% 
% \author{HKFoggyU
% \thanks{GitHub: \href{HKFoggyU}{https://github.com/HKFoggyU/hkust-thesis/}}}
%
% \date{v0.1 \\ Released 2021-10-21}
%
% \changes{v0.1}{2021/10/21}{Start development based on \textsc{NjuThesis} class}
%
% \maketitle
%
% \def\abstractname{Abstract}
% \begin{abstract}
% The \textsc{HkustThesis} class is intended for 
% typesetting dissertations with \hologo{LaTeX} for MPhil and PhD students in
% The Hong Kong University of Science and Technology (HKUST).
% \end{abstract}
%
% \vspace{2cm}
% \def\abstractname{Statement}
% \begin{abstract}
% This template is not officially related with HKUST.
% This template takes absolutely no responsibilty for any inconsistencies compared with the requirements by HKUST or department.
%
% This template is under development. Bugs or inconsistencies may happen. Issues and PRs are welcomed.
% \end{abstract}
%
% \clearpage
%
% \setcounter{tocdepth}{4}
% \tableofcontents
% \clearpage
%
% \EnableDocumentation
%^^A \DisableDocumentation
% 
% \begin{documentation}
%
% \section{使用方法}
%
% \subsection{File tree}
%
% \cls{hkustthesis} should have the following file tree:
%
% \begin{ctexexam}
%   \documentclass[<options>]{hkustthesis}
%   \hkustsetup { info = {<info>} }
%   \graphicspath{{figure/}}
%   \addbibresource{hkustthesis.bib}
%   \begin{document}
%   \maketitle
%   <abstract>
%   <preface>
%   \tableofcontents
%   \listoffigures
%   \listoftables
%   \mainmatter
%   <text>
%   \printbibliography[heading=bibintoc,title=References]
%   <acknowledgements>
%   \appendix
%   <appendix>
%   \end{document}
% \end{ctexexam}
%
% \end{documentation}
%
% \newpage
%
% \begin{implementation}
%
% \section{Implementation}
%
% |@@| 在 \pkg{l3docstrip} 中表示名空间,在删除注释生成格式文件时会被等号后的字段替换,譬如在本模板\pkg{hkustthesis}中 |@@=hkust|。
% 尖括号包裹的|<*class>||</class>|用来指定某段代码属于哪个文件。
%
%    \begin{macrocode}
%<@@=hkust>
%<*class>
%    \end{macrocode}
%
% \subsection{定义常量}
%
% \begin{macro}{\@@_define_name:nn}
% 用来定义默认名称的辅助函数。
%    \begin{macrocode}
\cs_new_protected:Npn \@@_define_name:nn #1#2
  { \tl_const:cn { c_@@_name_ #1 _tl } {#2} }
%    \end{macrocode}
% \end{macro}
%
% 默认名称。注意空格是忽略掉的。
%    \begin{macrocode}
\clist_map_inline:nn
  {
    { pdf_creator } { LaTeX~with~hkustthesis~class },
  }
  { \@@_define_name:nn #1 }
\clist_map_inline:nn
  {
    { keywords } { Keywords:~ },
  }
  { \@@_define_name:nn #1 }
%    \end{macrocode}
%
% \subsection{模板选项}
%
% 用于配置模板选项的宏包。
%    \begin{macrocode}
\RequirePackage{xparse,xtemplate,l3keys2e}
%    \end{macrocode}
%
% \begin{variable}{\l_@@_info_degree_tl,\l_@@_info_type_tl}
% 用于存储学位名称的变量,注意宏的命名,\verb|l|代表局部变量,\verb|g|代表全局变量
%    \begin{macrocode}
\tl_new:N \l_@@_info_degree_tl
\tl_new:N \l_@@_info_type_tl
%    \end{macrocode}
% \end{variable}
%
% \begin{variable}{\g_@@_latin_fontset_tl}
% 用于存储所使用字体名称的全局变量
%    \begin{macrocode}
\tl_new:N \g_@@_latin_fontset_tl
%    \end{macrocode}
% \end{variable}
%
% 学位信息的设置
%    \begin{macrocode}
\keys_define:nn { hkust }
{
%    \end{macrocode}
%
% \begin{macro}{customlatinfont}
% 定义字体选项
%    \begin{macrocode}
  customlatinfont   .choices:nn   =
  { gyre, macos, windows, none }
  { \tl_set_eq:NN \g_@@_latin_fontset_tl \l_keys_choice_tl },  
}
%    \end{macrocode}
% \end{macro}
%
% \begin{macro}{\ProcessKeysOptions}
% 在定义完全部设置以后从tex文件导言区输入参数
%    \begin{macrocode}
\ProcessKeysOptions { hkust }
%    \end{macrocode}
% \end{macro}
%
% \subsection{Personal Information}
% Input author's personal information
%    \begin{macrocode}
\keys_define:nn { hkust }
{
  info.meta:nn = { hkust / info } { #1 }
}
%    \end{macrocode}
%
%    \begin{macrocode}
\keys_define:nn { hkust / info }
{
%    \end{macrocode}
%
% \begin{macro}{info/degree}
% Degree: <PhD> or MPhil
%    \begin{macrocode}
  degree.tl_set:N = \l_@@_info_degree_tl,
%    \end{macrocode}
% \end{macro}
%
% \begin{macro}{info/type}
% Thesis Type: <thesis> or design
%    \begin{macrocode}
  type.tl_set:N = \l_@@_info_type_tl,
%    \end{macrocode}
% \end{macro}
%% \begin{macro}{info/title}
% Thesis title
%    \begin{macrocode}
  title             .tl_set:N = \l_@@_info_title_tl,
%    \end{macrocode}
% \end{macro}
%
% \begin{macro}{info/keywords}
% Keywords
%    \begin{macrocode}
keywords         .clist_set:N = \l_@@_info_keywords_clist,
%    \end{macrocode}
% \end{macro}
% 
% \begin{macro}{info/grade,info/student-id,info/author}
% 年级、学号、姓名
%    \begin{macrocode}
  grade             .tl_set:N = \l_@@_info_grade_tl,
  student-id        .tl_set:N = \l_@@_info_id_tl,
  author            .tl_set:N = \l_@@_info_author_tl,
  city              .tl_set:N = \l_@@_info_city_tl,
%    \end{macrocode}
% \end{macro}
% 
% \begin{macro}{info/school,info/department,info/major,info/field}
% 院系、专业、方向。
%    \begin{macrocode}
  school            .tl_set:N = \l_@@_info_school_tl,
  department        .tl_set:N = \l_@@_info_department_tl,
  major             .tl_set:N = \l_@@_major_tl,
  field             .tl_set:N = \l_@@_field_tl,
%    \end{macrocode}
% \end{macro}
%  
% \begin{macro}{info/supervisor,info/supervisor-title}
% Supervisor
%    \begin{macrocode}
  supervisor        .tl_set:N = \l_@@_info_supervisor_tl,
  supervisor-title  .tl_set:N = \l_@@_info_supervisor_title_tl,
%    \end{macrocode}
% \end{macro}
%
% \begin{macro}{info/co-supervisor,info/co-supervisor-title}
% Co-Supervisor
%    \begin{macrocode}
  co-supervisor      .tl_set:N = \l_@@_info_co_supervisor_tl,
  co-supervisor-title.tl_set:N = \l_@@_info_co_supervisor_title_tl,
%    \end{macrocode}
% \end{macro}
%
% \begin{macro}{info/submit-date,info/submit-date-long}
  % 提交日期
%    \begin{macrocode}
  submit-date       .tl_set:N = \l_@@_submit_date_tl,
  submit-date-long  .tl_set:N = \l_@@_submit_date_long_tl,
%    \end{macrocode}
% \end{macro}
%
% \begin{macro}{info/defend-date,info/chairman,info/depthead,info/reviewer}
% 答辩 TODO: 用clist处理答辩委员会成员名称
%    \begin{macrocode}
  defend-date       .tl_set:N = \l_@@_defend_date_tl,
  chairman          .tl_set:N = \l_@@_info_chairman_tl,
  depthead          .tl_set:N = \l_@@_info_depthead_tl,
  reviewer       .clist_set:N = \l_@@_info_reviewer_clist,
}
%    \end{macrocode}
% \end{macro}
%
%
% \begin{macro}{\hkustsetup}
% 定义用于设置个人信息的命令
%    \begin{macrocode}
\NewDocumentCommand \hkustsetup { m }
{ \keys_set:nn { hkust } { #1 } }
%    \end{macrocode}
% \end{macro}
%
% \subsection{载入文档类}
% 
% 使用\pkg{book}文档类。
%    \begin{macrocode}
\LoadClass[a4paper,twoside,UTF8,
%    \end{macrocode}
% 关于行距,\hologo{LaTeX}默认1.2行距,word默认行距是1.3,要求1.5倍word行距,故
% \[ 1.5\times\frac{1.3}{1.2} = 1.625\]
% TODO: For HKUST, the "line space" is set to 1.0 for abstract, footnote and quotations.
%    \begin{macrocode}
%  linespread=1.625,
  fontset=none,12pt]{book}
%    \end{macrocode}
%
% \subsection{载入宏包}
%
% 载入各种宏包。
% \pkg{emptypage}用于清除空白页的页码。
%    \begin{macrocode}
\RequirePackage
{
  geometry,
  caption,
  floatrow,
  setspace,
  lastpage,
  emptypage,
  fancyhdr,
}
\RequirePackage[explicit]{titlesec} % For typesetting titles of chap/sec/...
\RequirePackage[titles]{tocloft}
\RequirePackage[hyphens]{url} % generate better linebreaks in the url
%    \end{macrocode}
%
% 用于特定学科的包。
%    \begin{macrocode}
\RequirePackage{siunitx}            % 用于书写单位符号
\RequirePackage[version=4]{mhchem}  % 用于绘制分子式
\RequirePackage{physics}            % Physics package
\RequirePackage{braket}             % for Dirac notation
%    \end{macrocode}
%
% 用于生成可以被插入书签的LaTeX logo
% TODO: 使用hologo创建|latex{}|命令
%    \begin{macrocode}
\RequirePackage{hologo} 
%    \end{macrocode}
%
%    \begin{macrocode}
% \RequirePackage{needspace} % prevent page break after sectioning
% \RequirePackage{xspace} % Better print trailing whitespace
%    \end{macrocode}
%
% \pkg{ulem} for underline with "\uline"
%    \begin{macrocode}
\RequirePackage{ulem}
%    \end{macrocode}
% \pkg{amsmath}必须在\pkg{unicode-math}前加载。
% \pkg{unicode-math}指定了\hologo{XeTeX}和\hologo{LuaTeX}下所使用的数学字体。
% 用于配置数学环境的\pkg{mathtools}会与\pkg{unicode-math}发生冲突,此处手动消除其警告。
%    \begin{macrocode}
\RequirePackage{amsmath,amsthm,mathtools,thmtools}
\RequirePackage[
    warnings-off={
        mathtools-colon,
        mathtools-overbracket}
        ]{unicode-math}
%    \end{macrocode}
%
% 配置图片、表格、代码、列表环境
%    \begin{macrocode}
\RequirePackage{graphicx,subcaption,wrapfig,tikz}
\DeclareGraphicsExtensions{.pdf,.eps,.jpg,.png}
\RequirePackage{booktabs,multirow,multicol,listings,enumitem}
%    \end{macrocode}
%
% 必须以该顺序加载以下两个关于引用的包。
%    \begin{macrocode}
\RequirePackage[
  hidelinks,
  bookmarksnumbered=true,
  psdextra=true,
  unicode=true]{hyperref}
\RequirePackage[capitalise,nameinlink,noabbrev]{cleveref}
%    \end{macrocode}
%
% 生成 "Lorem ipsum"
%    \begin{macrocode}
\RequirePackage{blindtext} 
%    \end{macrocode}
%
% \subsection{字体设置}
%
%    \begin{macrocode}
\RequirePackage{fontspec} 
%    \end{macrocode}
%
% \subsubsection{操作系统检测}
%
% \begin{variable}{\g_@@_load_system_fontset_bool}
% 定义用于判断是否需要载入系统预装字体的变量。
%    \begin{macrocode}
\bool_new:N \g_@@_load_system_fontset_bool
%    \end{macrocode}
% \end{variable}
%
% 判断用户是否自定义了中英文字体。如果其中任意一种未被定义,
% 则使用系统预装字体覆盖字体选项。
%    \begin{macrocode}
\tl_if_empty:NTF \g_@@_latin_fontset_tl
  { \bool_gset_true:N \g_@@_load_system_fontset_bool }  
  { }
%    \end{macrocode}
%
% 进行系统检测。
% 检测 Windows 的命令由\pkg{l3kernal}提供,
% 检测 macOS 的命令 modified from \pkg{ctex},
% 以特定字体判断 macOS 系统。
%    \begin{macrocode}
\cs_new_protected:Npn \@@_if_platform_macos:TF
  { \file_if_exist:nTF { /System/Library/Fonts/Menlo.ttc } }
%    \end{macrocode}
% 这两种情况外的系统被判断为 Linux,一律使用自由字体。
%    \begin{macrocode}
\bool_if:NT \g_@@_load_system_fontset_bool
{
  \sys_if_platform_windows:TF
  {
    \tl_set:Nn \g_@@_latin_fontset_tl { windows }
  }
  {
    \@@_if_platform_macos:TF
    {
      \tl_set:Nn \g_@@_latin_fontset_tl { macos }
    }
    {
      \tl_set:Nn \g_@@_latin_fontset_tl { gyre }
    }
  }
}
%    \end{macrocode}
%
% \subsubsection{定义英文字库}
%
% 接下来逐个定义所需要使用的字库。
%
% \begin{macro}{\@@_load_latin_font_windows:}
% Windows 西文字体
%    \begin{macrocode}
\cs_new_protected:Npn \@@_load_latin_font_windows:
{
  \setmainfont{Times~New~Roman}
  \setsansfont{Arial}
  \setmonofont{Courier~New}[Scale=MatchLowercase]
}
%    \end{macrocode}
% \end{macro}
%
% \begin{macro}{\@@_load_latin_font_macos:}
% macOS 西文字体。
%    \begin{macrocode}
\cs_new_protected:Npn \@@_load_latin_font_macos:
{
  \setmainfont{Times~New~Roman}
  \setsansfont{Arial}
  \setmonofont{Menlo}[Scale=MatchLowercase]
}
%    \end{macrocode}
% \end{macro}
%
% \begin{macro}{\@@_load_latin_font_gyre:}
% 开源的 gyre 西文字体。
%    \begin{macrocode}
\cs_new_protected:Npn \@@_load_latin_font_gyre:
{
  \setmainfont{texgyretermes}[
    Extension=.otf,
    UprightFont=*-regular,
    BoldFont=*-bold,
    ItalicFont=*-italic,
    BoldItalicFont=*-bolditalic]
  \setsansfont{texgyreheros}[
    Extension=.otf,
    UprightFont=*-regular,
    BoldFont=*-bold,
    ItalicFont=*-italic,
    BoldItalicFont=*-bolditalic]
  \setmonofont{texgyrecursor}[
    Extension=.otf,
    UprightFont=*-regular,
    BoldFont=*-bold,
    ItalicFont=*-italic,
    BoldItalicFont=*-bolditalic,
    Scale=MatchLowercase,
    Ligatures=CommonOff]
}
%    \end{macrocode}
% \end{macro}
%
% \subsubsection{载入指定字库}
%
% 载入字体命令。
%    \begin{macrocode}
\cs_new_protected:Npn \@@_load_font:
{
  \use:c { @@_load_latin_font_ \g_@@_latin_fontset_tl : }
}
%    \end{macrocode}
%
% 载入设置的字体。
%    \begin{macrocode}
\@@_load_font:
%    \end{macrocode}
%
% 设置数学字体 (XITS, 或者 \href{https://www.stixfonts.org}{STIX}, 与 Times New Roman 最为相近)
%    \begin{macrocode}
% \setmathfont{STIXTwoMath-Regular}[Extension = .otf]
\setmathfont{XITSMath-Regular}[
  BoldFont = XITSMath-Bold,
  Extension = .otf]
\setmathfont{latinmodern-math.otf}[range={cal,bb,frak}]
%    \end{macrocode}
%
% \subsection{页面布局}
%
% \subsubsection{页边距}
%
% 使用\pkg{geometry}设置页边距。
%    \begin{macrocode}
\geometry{
  vmargin    = 2.5 cm,
  hmargin    = 2.5 cm,
}
%    \end{macrocode}
%
% \subsubsection{页眉页脚}
% 
%    \begin{macrocode}
\fancypagestyle{hkustplain}{
   \fancyhead{}               
   \fancyfoot[C]{\thepage}
}
%    \end{macrocode}
%
% TODO: 研究生页眉页脚 
%
% 载入页眉页脚设置。此处\tn{flushbottom}是为了防止目录页出现underfull \tn{vbox}信息。
%    \begin{macrocode}
\tl_set:Nn \headrulewidth {0pt}
\tl_set:Nn \footrulewidth {0pt}
\AtBeginDocument{\pagestyle{hkustplain}\flushbottom}
%    \end{macrocode}
%
% \subsection{章节标题格式}
% 
%    \begin{macrocode}
\titleformat
  {\chapter}
  [display]
  {\centering\fontsize{20pt}{24pt}\bfseries\selectfont}
  {\MakeUppercase{\chaptertitlename}~\thechapter}
  {10pt}
  {\MakeUppercase{#1}}
%    \end{macrocode}
%
% \subsection{目录格式}
% 使用\pkg{tocloft}定制目录文字格式。
%    \begin{macrocode}
\cftsetpnumwidth{2em}
\renewcommand{\cftchapleader}{\cftdotfill{\cftchapdotsep}}
\renewcommand{\cftchapdotsep}{\cftdotsep}
\renewcommand{\cftchapfont}{\fontsize{12pt}{14pt}\bf\rmfamily\selectfont}
\setlength{\cftsecindent}{2em}
\setlength{\cftsubsecindent}{52pt}
\setlength{\cftsubsecnumwidth}{2em}
%    \end{macrocode}
%
% \begin{macro}{\tableofcontents}
  % 重定义目录命令,修改标题格式并插入书签。
%    \begin{macrocode}
\renewcommand\tableofcontents{%
  \cleardoublepage
  \raggedbottom
  \chapter*{\MakeUppercase{Table~of~Contents}}%
  \addcontentsline{toc}{chapter}{Table~of~Contents}
  \@starttoc{toc}%
}
%    \end{macrocode}
% \end{macro}
%
% \begin{macro}{\listoffigures}
% 重定义插图目录命令,修改标题格式并插入书签。
%    \begin{macrocode}
\renewcommand\listoffigures{%
  \cleardoublepage
  \chapter*{\MakeUppercase{List~of~Figures}}%
  \addcontentsline{toc}{chapter}{List~of~Figures}
  \@starttoc{lof}%
}
%    \end{macrocode}
% \end{macro}
%
% \begin{macro}{\listoftables}
% 重定义表格目录命令,修改标题格式并插入书签。
%    \begin{macrocode}
\renewcommand\listoftables{%
  \cleardoublepage
  \begingroup
  \chapter*{\MakeUppercase{List~of~Tables}}%
  \addcontentsline{toc}{chapter}{List~of~Tables}
  \@starttoc{lot}%
}
%    \end{macrocode}
% \end{macro}
%
% \subsection{前言致谢}
% 
% \begin{environment}{preface}
% 单独制作的前言页。
%    \begin{macrocode}
\NewDocumentEnvironment{preface}{}
{%
  \chapter*{Preface}
  \addcontentsline{toc}{chapter}{Preface}
}{\cleardoublepage}
%    \end{macrocode}
% \end{environment}
%
% \begin{environment}{acknowledgements}
% 单独制作的致谢页。
%    \begin{macrocode}
\NewDocumentEnvironment{acknowledgements}{}
{%
  \chapter*{Acknowledgements}
  \addcontentsline{toc}{chapter}{Acknowledgements}
}{\cleardoublepage}
%    \end{macrocode}
% \end{environment}
%
% \begin{macro}{\paperlist}
% 发表文章目录与合作文章目录。
%    \begin{macrocode}
\NewDocumentCommand\paperlist { t* m }
{  
  \begin{refsection} 
    \nocite{#2}
    \IfBooleanTF {#1}
    { \printbibliography[heading=subbibliography,title=Conference~Publications] }
    { \printbibliography[heading=subbibliography,title=Journal~Publications] }
  \end{refsection}
}
%    \end{macrocode}
% \end{macro}
%
% \subsection{参考文献}
% 
% biblatex设置
%    \begin{macrocode}
\RequirePackage[
    style=ieee,
    %style=numeric-comp,
    %citestyle=authortitle-icomp,
    % citestyle=numeric-comp,
    %bibstyle=authoryear,
    % bibstyle=numeric,
    sorting=none,
    %sorting=nyt,
    %sortcites=true,
    %autocite=footnote,
    backend=biber, % Compile the bibliography with biber
    hyperref=true,
    backref=false,
    citecounter=true,
    pagetracker=true,
    citetracker=true,
    ibidtracker=context,
    autopunct=true,
    autocite=plain,
    % gbpub=false,         % Uncomment if you do NOT want '[S.l. : s.n.]' 
                           % in reference entries, GitHub Issue (#47)
    % gbnamefmt=lowercase, % Uncomment if you do NOT want uppercase author 
                           % names in reference entries, GitHub Issue (#23)
]{biblatex}
%    \end{macrocode}
%
% 忽略不需要的文献信息。
%    \begin{macrocode}
\AtEveryBibitem{
    \clearfield{abstract}
    \clearfield{issn}
    \clearfield{isbn}
    \clearfield{archivePrefix}
    \clearfield{arxivId}
    \clearfield{pmid}
    \clearfield{eprint}
    \ifentrytype{online}{}{\ifentrytype{misc}{}{\clearfield{url}}}
    % \ifentrytype{book}{\clearfield{doi}}{}
}
%    \end{macrocode}
%
% \subsection{引用}
% 
% 修改标签名称。默认在名称后面添加空格,删除公式编号的括号
%    \begin{macrocode}
\crefdefaultlabelformat{#2#1#3\,}

\crefname{figure}{图}{图}
\crefname{table}{表}{表}
% \crefname{equation}{公式}{公式}
\crefformat{equation}{公式~#2#1#3~}

\crefformat{chapter}{第#2#1#3章}
\crefformat{section}{第~#2#1#3~节}
\crefformat{subsection}{第~#2#1#3~小节}
\crefformat{subsubsection}{第~#2#1#3~小节}
\crefname{appendix}{附录}{附录}

% \crefname{definition}{定义}{定义}
% \crefname{axiom}{公理}{公理}
% \crefname{property}{性质}{性质}
% \crefname{proposition}{命题}{命题}
% \crefname{lemma}{引理}{引理}
% \crefname{corollary}{推论}{推论}
% \crefname{remark}{注解}{注解}
% \crefname{condition}{条件}{条件}
% \crefname{conclusion}{结论}{结论}
% \crefname{assumption}{假设}{假设}
%    \end{macrocode}
%
% CHANGELOG: Get rid of all ctex packages
% \pkg{hyperref} 
%    \begin{macrocode}
\cs_new_protected:Npx \@@_gadd_ltxhook:nn #1
  { \hook_gput_code:nnn {#1} { \c_novalue_tl } }
\cs_new_protected:Npn \@@_at_end_preamble:n
  { \@@_gadd_ltxhook:nn { begindocument/before } }
\@@_at_end_preamble:n
{
  \hypersetup
    {
      pdftitle    = \l_@@_info_title_tl,
      pdfauthor   = \l_@@_info_author_tl,
      pdfkeywords = \l_@@_info_keywords_clist,
      pdfcreator  = \c_@@_name_pdf_creator_tl,
      pdfsubject  = to~her,
      pdfproducer = love
    }
}
%    \end{macrocode}
%
% \subsection{图表浮动体}
% 
% \subsubsection{图片表格}
% 
% 图表位置调整
%    \begin{macrocode}
\floatsetup[figure]{ % Captions for figures
    capposition=bottom,%
    margins=centering,%
    floatwidth=\textwidth%
}
\floatsetup[table]{ % Captions for tables
    capposition=above,%
    margins=centering,%
    floatwidth=\textwidth%
}
%    \end{macrocode}
% 
% 图表标题样式
%    \begin{macrocode}
\DeclareCaptionFont{capfont}{\fontsize{12pt}{14pt}\bf\rmfamily\selectfont}
\captionsetup{
  font=small,%
  labelfont=capfont,
    textfont=capfont,
    strut=no,%
    hypcap=true, % Links point to the top of the figure
    % indention=0pt, % Suppress indentation
    % % parindent=0pt, % Suppress space between paragraphs
    aboveskip=6pt, % Increase the space between the figure and the caption
    belowskip=6pt, % Increase the space between the caption and the table
}
%    \end{macrocode}
%
% \subsubsection{代码}
% 
% 代码样式
%    \begin{macrocode}
\floatsetup[lstlisting]{ % Captions for lstlistings
    capposition=above,%
    margins=centering,%
    floatwidth=\textwidth%
}
\lstset{
    basicstyle=\ttfamily\linespread{1}\small\selectfont,
    keywordstyle=\bfseries,% use bold style for keywords
    commentstyle=\rmfamily\itshape,% use italic style for comments
    stringstyle=\ttfamily,% 字符串风格
    flexiblecolumns,% ?
    numbers=left,% left-aligned numbering
    showspaces=false,% hide markers for spaces
    showstringspaces=false,
    captionpos=t,% place the caption at the top
    % frame=lrtb,% show all four sides of the frame
    % linewidth=.8\textwidth,
    % breakatwhitespace=true,
    breaklines=true,
    xleftmargin=2em,xrightmargin=2em,% set the width of the code environment
}
%    \end{macrocode}
%
%    \begin{macrocode}
\lstdefinestyle{LaTeX}{
  language=TeX,
  morekeywords={
    begin, caption, label, mathrm, frac, 
    toprule, midrule, bottomrule, includegraphics}
}
%    \end{macrocode}
%
% \subsubsection{列表}
% 
% 列表环境
%    \begin{macrocode}
\renewcommand{\labelitemi}{\tiny$\blacktriangleright$}
\renewcommand{\labelitemii}{\textbullet}

\setlist[itemize]{noitemsep}
\setlist[enumerate]{noitemsep}
\setlist[description]{noitemsep}
%    \end{macrocode}
% 
% \subsection{定理环境}
%
% \begin{macro}{\mathbi}
% Math bold italic letters
%    \begin{macrocode}
\NewDocumentCommand\mathbi{m}{\textbf{\em #1}}
%    \end{macrocode}
% \end{macro}
%
%    \begin{macrocode}
\declaretheoremstyle[
  %spaceabove=.5\thm@preskip,
  %spacebelow=.5\thm@postskip,
  headfont=\fontsize{12pt}{14pt}\bf\rmfamily\selectfont,%\scshape,
  notefont=\fontsize{12pt}{14pt}\rmfamily\selectfont,% notebraces={ (}{)},
  bodyfont=\fontsize{12pt}{14pt}\rmfamily\selectfont,
  %headformat={\NAME\space\NUMBER\space\NOTE},
  headpunct={},
  %postheadspace={.5em plus .1em minus .1em},
  %prefoothook={\hfill\qedsymbol}
]{hkustthm}

\theoremstyle{hkustthm}
%    \end{macrocode}
%
% 修改证明环境标题 
%    \begin{macrocode}
\let\oldproofname=\proofname
\renewcommand*{\proofname}
  {\fontsize{12pt}{14pt}\rmfamily\selectfont{\oldproofname}} 
%    \end{macrocode}
%
% 
%    \begin{macrocode}
\declaretheorem[
    name=算法,
    style=hkustthm,
    refname={算法,算法},
    Refname={算法,算法},
    % numberwithin=section,
]{algorithm}
\declaretheorem[
    name=假设,
    style=hkustthm,
    refname={假设,假设},
    Refname={假设,假设},
    % numberwithin=section,
]{assumption}
\declaretheorem[
    name=公理,
    style=hkustthm,
    refname={公理,公理},
    Refname={公理,公理},
    % numberwithin=section,
]{axiom}
\declaretheorem[
    name=结论,
    style=hkustthm,
    refname={结论,结论},
    Refname={结论,结论},
    % numberwithin=section,
]{conclusion}
\declaretheorem[
    name=条件,
    style=hkustthm,
    refname={条件,条件},
    Refname={条件,条件},
    % numberwithin=section,
]{condition}
\declaretheorem[
    name=推论,
    style=hkustthm,
    refname={推论,推论},
    Refname={推论,推论},
    % numberwithin=section,
]{corollary}
\declaretheorem[
    name=定义,
    style=hkustthm,
    refname={定义,定义},
    Refname={定义,定义},
    % numberwithin=section,
]{definition}
\declaretheorem[
        name=例,
        style=hkustthm,
        refname={例,例},
        Refname={例,例},
        % numberwithin=section,
]{example}
\declaretheorem[
    name=引理,
    style=hkustthm,
    refname={引理,引理},
    Refname={引理,引理},
    % numberwithin=section,
]{lemma}
\declaretheorem[
    name=性质,
    style=hkustthm,
    refname={性质,性质},
    Refname={性质,性质},
    % numberwithin=section,
]{property}
\declaretheorem[
    name=命题,
    style=hkustthm,
    refname={命题,命题},
    Refname={命题,命题},
    % numberwithin=section,
]{proposition}
\declaretheorem[
    name=注解,
    style=hkustthm,
    refname={注解,注解},
    Refname={注解,注解},
    % numberwithin=section,
]{remark}
\declaretheorem[
    name=定理,
    style=hkustthm,
    refname={定理,定理},
    Refname={定理,定理},
    numberwithin=section,
]{theorem}
%    \end{macrocode}
%
% \subsection{封面}
%
% \subsubsection{内部命令}
%
% \begin{variable}{\l_@@_info_supervisor_title_name_tl,l_@@_info_co_supervisor_title_name_tl,\l_@@_info_supervisor_title_name_full_tl}
% 用于存储导师姓名加职称的变量,旧版编译器不支持字符串中含有|\hspace{.5em}|这样的空白空间命令
%    \begin{macrocode}
\tl_new:N \l_@@_info_supervisor_title_name_tl
\tl_new:N \l_@@_info_co_supervisor_title_name_tl
\tl_new:N \l_@@_info_supervisor_title_name_full_tl
%    \end{macrocode}
% \end{variable}
%
% 拼合导师的职称和姓名。
%    \begin{macrocode}
\tl_set:Nn \l_@@_info_supervisor_title_name_tl
{
  \l_@@_info_supervisor_title_tl\ 
  \l_@@_info_supervisor_tl
}
\tl_set:Nn \l_@@_info_co_supervisor_title_name_tl
{
  \l_@@_info_co_supervisor_title_tl\ 
  \l_@@_info_co_supervisor_tl
}
% 拼合双导师的姓名和职称。
\tl_set:Nn \l_@@_info_supervisor_title_name_full_tl
{
  \l_@@_info_supervisor_title_name_tl \
  \l_@@_info_co_supervisor_title_name_tl
}
%    \end{macrocode}
%
% \begin{variable}{\l_@@_name_degree_tl}
% 用于存储学位名称的变量
%    \begin{macrocode}
\tl_new:N \l_@@_name_degree_tl
%    \end{macrocode}
% \end{variable}
%
% 判断学位进行命令定义
%    \begin{macrocode}
\tl_const:Nn \l_@@_name_diploma_tl { RPG }
% 研究生学位名称
\str_if_eq:NNTF { \l_@@_info_degree_tl } { PhD }
{ \tl_set:Nn \l_@@_name_degree_tl { Doctor~of~Philosophy } }
{ \str_if_eq:NNTF { \l_@@_info_degree_tl } { MPhil }
  {\tl_set:Nn \l_@@_name_degree_tl { Master~of~Philosophy } }
  {\tl_set:Nn \l_@@_name_degree_tl { Degree~Unspported } }
}
%    \end{macrocode}
%
% \begin{macro}{\l_@@_name_diploma_tl}
% Thesis paper or design
%    \begin{macrocode}
\str_if_eq:NNTF { \l_@@_info_type_tl } { thesis } 
{ \tl_const:Nn \l_@@_info_type_tl_name 
    { \l_@@_name_diploma_tl~thesis~paper } }
{ \tl_const:Nn \l_@@_info_type_tl_name 
    { \l_@@_name_diploma_tl~thesis~design }}
%    \end{macrocode}
% \end{macro}
%
% \subsubsection{绘制封面}
% 
% \begin{macro}{\@@_print_cover:}
% Title page for thesis
%    \begin{macrocode}
\cs_new_protected:Npn \@@_print_cover:
{
  \thispagestyle{plain}
  \begin{center}
    \rmfamily\fontsize{12pt}{18pt}\selectfont\l_@@_info_title_tl \\
    \setlength{\parskip}{18pt plus 1pt minus 1pt}
    \phantom{1}\\
    \phantom{2}\\
    \phantom{3}\\
    \phantom{4}\\
    \phantom{5}\\
    \rmfamily\fontsize{12pt}{18pt}\selectfont{by} \\
    \phantom{1}\\
    \rmfamily\fontsize{12pt}{14pt}\selectfont\l_@@_info_author_tl
    \phantom{1}\\
    \phantom{2}\\
    \phantom{3}\\
    \phantom{4}\\
    \phantom{5}\\
    \normalsize\rmfamily{%
      A~Thesis~Submitted~to\\
      The~Hong~Kong~University~of~Science~and~Technology\\
      in~Partial~Fulfilment~of~the~Requirements~for\\
      the~Degree~of~{\l_@@_name_degree_tl}\\
      in~{\l_@@_major_tl}
    }
    \phantom{1}\\
    \phantom{2}\\
    \phantom{3}\\
    \normalfont\normalsize\l_@@_submit_date_tl,~\l_@@_info_city_tl
  \end{center}
  \addcontentsline{toc}{chapter}{Title~Page}
  \normalfont
  \cleardoublepage
}
%    \end{macrocode}
% \end{macro}
% 
% \begin{macro}{\maketitle}
% 重定义maketitle生成封面
%    \begin{macrocode}
\tl_set:Nn \maketitle {\@@_print_cover:}
%    \end{macrocode}
% \end{macro}
%
% \subsection{摘要绘制}
%
% \begin{macro}{\@@_print_keywords:}
% Keywords
%    \begin{macrocode}
\cs_new_protected:Npn \@@_print_keywords:
{
  \par\vspace{2ex}
  \bgroup
    \noindent
    \c_@@_name_keywords_tl
    \clist_use:Nn \l_@@_info_keywords_clist {,~} 
    \par
  \egroup
}
%    \end{macrocode}
% \end{macro}
%   
% \begin{macro}{\@@_print_abstractg:}
% Output Abstract
%    \begin{macrocode}
\cs_new_protected:Npn \@@_print_abstractg:
{
  \cleardoublepage
  \thispagestyle{plain}
  \phantomsection
  \addcontentsline{toc}{chapter}{Abstract}
  {
    \bgroup
    \fontsize{12pt}{14pt}\selectfont
    \begin{center}
    \fontsize{12pt}{12pt}\selectfont
    \l_@@_info_title_tl \\
    by~\l_@@_info_author_tl \\
    \l_@@_info_department_tl \\
    The Hong Kong University of Science and Technology
    Abstract
    \end{center}
    \egroup
  }
  \fontsize{12pt}{14pt}\selectfont\par%
}
%    \end{macrocode}
% \end{macro}
%
%    \begin{macrocode}
\NewDocumentEnvironment{abstract} {}
{%
  \@@_print_abstractg:}{\@@_print_keywords:\cleardoublepage
}
%    \end{macrocode}
%
% \subsection{Authorization}
% \begin{macro}{\@@_print_authorization_g:}
% Output authorization
%    \begin{macrocode}
\cs_new_protected:Npn \@@_print_authorization_g:
{
  \thispagestyle{plain}
  \phantomsection
  \addcontentsline{toc}{chapter}{Authorization}
  {
    \fontsize{12}{18}\selectfont
    {
    \begin{center}
    \bf\uline{Authorization}
    \end{center}
    }
    I~hereby~declare~that~I~am~the~sole~author~of~the~thesis.\\

    I~authorize~the~Hong~Kong~University~of~Science~and~Technology~to~lend~this~thesis~to~other~institutions~or~individuals~for~the~purpose~of~scholarly~research.\\

    I~further~authorize~the~Hong~Kong~University~of~Science~and~Technology~to~reproduce~the~thesis~by~photocopying~or~by~other~means,~in~total~or~in~part,~at~the~request~of~other~institutions~or~individuals~for~the~purpose~of~scholarly~research.
    {
    \begin{center}
    \uline{\phantom{Author signature}}\\
    \l_@@_info_author_tl \\
    \l_@@_submit_date_long_tl
    \end{center}
    }
 }
  \fontsize{12pt}{14pt}\selectfont\par%
}
%    \end{macrocode}
% \end{macro}
%
%    \begin{macrocode}
\NewDocumentCommand\authorization{}{\@@_print_authorization_g:\cleardoublepage}
%    \end{macrocode}
%
% \subsection{Signature~Page}
% \begin{macro}{\@@_print_signaturepage_g:}
% Output signature page
%    \begin{macrocode}
\cs_new_protected:Npn \@@_print_signaturepage_g:
{
  \thispagestyle{plain}
  \phantomsection
  \addcontentsline{toc}{chapter}{Signature~Page}
  {
    {
    \begin{center}
    \fontsize{12}{18}\selectfont
    \l_@@_info_title_tl \\
    by \\
    \l_@@_info_author_tl \\
    This~is~to~certify~that~I~have~examined~the~above~{\l_@@_info_degree_tl~}~thesis\\
    and~have~found~that~it~is~complete~and~satisfactory~in~all~respects,\\
    and~that~any~and~all~revisions~required~by\\
    the~thesis~examination~committee~have~been~made.\\
    \uline{\phantom{Supervisor name}} \\
    \l_@@_info_supervisor_title_name_tl,~Thesis~Supervisor \\
    \tl_if_empty:NTF \l_@@_info_co_supervisor_tl
    {}
    { \uline{\phantom{Co-Supervisor name}} \\
      \l_@@_info_co_supervisor_title_name_tl,~Co~Supervisor \\
    }
    \uline{\phantom{Department Head name}} \\
    \l_@@_info_depthead_tl,~Head~of~Department\\
    \l_@@_info_department_tl \\
    \l_@@_submit_date_long_tl
    \end{center}
    }
  }
  \fontsize{12pt}{14pt}\selectfont\par%
}
%    \end{macrocode}
% \end{macro}
%
%    \begin{macrocode}
\NewDocumentCommand\signaturepage{}{\@@_print_signaturepage_g:\cleardoublepage}
%</class>
%    \end{macrocode}
%%
% \end{implementation}
%
